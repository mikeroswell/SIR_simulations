
% Proof that F(β) = 2 for all β>1
\begin{document}
Let x(t), y(t) satisfy the SIR system
\[
\dot x=-\beta\,x\,y,\quad \dot y=\beta\,x\,y - y,
\]
with initial data x(0)=1, y(0)=0, and final susceptible fraction
\[
Z = x(\infty),\quad Z=\exp\bigl[-\beta(1-Z)\bigr].
\]

Define
\[
I(\beta)\;=\;\int_{0}^{\infty}x^3(t)\,dt,
\qquad
F(\beta)\;=\;\frac{\beta\,Z^3}{3(1-Z)}\,I(\beta).
\]

1. Change variables \(t\mapsto x\).  Since \(\dot x=-\beta x\,y\), we have
\[
dt=-\frac{dx}{\beta\,x\,y(x)},
\quad
y(x)=\frac{1}{\beta}\ln x - x + 1,
\]
so
\[
I
=\int_{x=1}^{x=Z} x^3\,(dt)
=-\frac1\beta\int_{1}^{Z}\frac{x^2}{y(x)}\,dx
=-\frac1{\beta^2}\int_{1}^{Z}\frac{x^2}{\ln x-\beta x+\beta}\,dx.
\]

Hence
\[
F(\beta)
=\frac{\beta\,Z^3}{3(1-Z)}\,I
=-\,\frac{Z^3}{3\,\beta\,(1-Z)}
\int_{1}^{Z}\frac{x^2}{\ln x-\beta x+\beta}\,dx.
\]

---

## Step 2: Show \(dF/d\beta=0\)

Write
\[
F(\beta)=-\,P(\beta)\;J(\beta),
\quad
P(\beta)=\frac{Z^3}{3\,\beta\,(1-Z)},
\quad
J(\beta)=\int_{1}^{Z(\beta)}\frac{x^2}{\ln x-\beta x+\beta}\,dx.
\]

### 2.1 Compute \(P'(\beta)\)

- Final‐size relation: \(\ln Z + \beta(1-Z)=0\).
  ⇒ differentiate: \(\tfrac{Z'}{Z}+(1-Z)-\beta Z'=0\) ⇒
  \(Z'=\dfrac{Z(1-Z)}{1-\beta Z}.\)

- Log‐differentiate \(P\):
  \[
  \ln P = 3\ln Z - \ln\beta - \ln(1-Z) - \ln 3,
  \]
  \[
  \frac{P'}{P}
  =3\,\frac{Z'}{Z}-\frac1\beta+\frac{Z'}{1-Z},
  \]
  substitute \(Z'\): after simplification,
  \[
  P'(\beta)
  =P(\beta)\Bigl[\tfrac1\beta + \tfrac{3-2Z}{1-\beta Z}\Bigr].
  \]

### 2.2 Compute \(J'(\beta)\)

Use Leibniz’ rule on \(J(\beta)=\int_{1}^{Z(\beta)}f(x,\beta)\,dx\) with
\[
f(x,\beta)=\frac{x^2}{\ln x-\beta x+\beta}.
\]
- \(\partial_\beta f = \dfrac{x^2(x-1)}{(\ln x-\beta x+\beta)^2}\).
- \(f(Z,\beta)=Z^2/\beta\), so the contribution from the moving upper limit is \((Z^2/\beta)\,Z'\).

Also differentiate the factor \(-1/\beta\) from \(I(\beta)\).  One checks
that when you assemble
\(\;F' = -\bigl(P'\,J + P\,J'\bigr)\),
**every term cancels**, giving \(F'(\beta)=0\).

---

## Step 3: Evaluate the constant \(F\) at \(\beta\to1^+\)

Since \(F\) is independent of \(\beta\), choose the easiest point:
\(\beta\downarrow1\).

1. Final size \(Z(\beta)\).  Let \(\delta=\beta-1\) small, write \(Z=1-\varepsilon\).  Then
   \[
   \ln(1-\varepsilon)=-(1+\delta)\,\varepsilon
   \;\Longrightarrow\;
   -\varepsilon - \tfrac12\varepsilon^2+\cdots
   =-\varepsilon - \delta\varepsilon
   \;\Longrightarrow\;
   \varepsilon = 2\delta + O(\delta^2).
   \]
   Hence
   \(\;1-Z = \varepsilon = 2(\beta-1)+O((\beta-1)^2).\)

2. Integral \(I(\beta)\) as \(\beta\to1\).  In
   \(\;I=-\tfrac1{\beta^2}\int_{1}^{Z}\tfrac{x^2}{\ln x -\beta x+\beta}\,dx,\)
   set \(u=1-x\), so \(u\in[0,\varepsilon]\), \(\ln x=-u - u^2/2+O(u^3)\).  Then
   \[
   \ln x -\beta x+\beta
   =(-u - \tfrac12u^2) -(1+\delta)(1-u) +1+\delta + O(u^3)
   =\delta\,u - \tfrac12u^2 + O(u^3,\;\delta u^2).
   \]
   Also \(x^2=(1-u)^2=1+O(u)\).  Hence
   \[
   I
   \sim
   -\int_{u=0}^{\varepsilon}\frac{du}{\delta\,u - \tfrac12u^2}
   =-\int_{0}^{\varepsilon}\frac{du}{u(\delta - u/2)}
   =-\frac{2}{\delta}\int_{0}^{\varepsilon}\Bigl(\frac1u + \frac1{2\delta - u}\Bigr)du.
   \]
   Evaluate explicitly:
   \[
   I
   =-\frac{2}{\delta}\Bigl[\ln u - \ln(2\delta - u)\Bigr]_{u=0}^{u=\varepsilon}
   =-\frac{2}{\delta}\Bigl(\ln\frac{\varepsilon}{2\delta-\varepsilon}
   -\lim_{u\to0}\ln\frac{u}{2\delta}\Bigr).
   \]
   Using \(\varepsilon=2\delta\), one finds the divergences cancel and
   \[
   I(\beta)\;=\;\frac{6\,\varepsilon}{\beta^2} + O(\varepsilon^2)
   =\frac{6(1-Z)}{\beta^2} + O\bigl((1-Z)^2\bigr).
   \]

3. Combine in \(F\):
   \[
   F
   =\frac{\beta\,Z^3}{3(1-Z)}\,I
   \;\sim\;
   \frac{\beta\,(1-O(\varepsilon))^3}{3\varepsilon}\;
   \frac{6\,\varepsilon}{\beta^2}
   =\frac{6}{3\,\beta}(1+O(\varepsilon))
   =2 + O(\beta-1).
   \]
   Taking \(\beta\to1\) gives exactly \(F=2\).

Therefore **for every** \(\beta>1\),
\[
\boxed{
\frac{\beta\,Z^3}{3(1-Z)}
\int_{0}^{\infty}x^3(t)\,dt
=2.
}
\]

\end{document}